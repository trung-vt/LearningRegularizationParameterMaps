\documentclass[12pt]{article}
\usepackage{xeCJK}
\usepackage{amssymb,tikz,pdftexcmds,xparse}
\usepackage{url}
\usepackage{titlesec}

\usepackage{indentfirst}

\usepackage[utf8]{inputenc}
\usepackage[english]{babel}
\usepackage{biblatex}
\addbibresource{references.bib}



% Language setting
% Replace `english' with e.g. `spanish' to change the document language
\usepackage[english]{babel}

% Set page size and margins
% Replace `letterpaper' with `a4paper' for UK/EU standard size
\usepackage[letterpaper,top=2cm,bottom=2cm,left=2cm,right=2cm,marginparwidth=1.75cm]{geometry}

% Useful packages
\usepackage{amsmath}
\usepackage{graphicx}
\usepackage[colorlinks=true, allcolors=blue]{hyperref}
\usepackage{listings}
\usepackage{color}
\usepackage{minted}

% https://www.overleaf.com/learn/latex/Positioning_images_and_tables#Basic_positioning
% To position the image to the centre
\usepackage[export]{adjustbox}  

\usepackage{xcolor}
\usepackage{xparse}
\usepackage{blindtext}
\usepackage{hyperref}   % For hyperlinks

\usemintedstyle{manni}

\NewDocumentCommand{\codeword}{v}{%
% \texttt{\textcolor{blue}{#1}}%
\texttt{\textcolor{black}{#1}}%
% \mint{html}|v|%
}

% \lstset{language=C,keywordstyle={\bfseries \color{blue}}}

\definecolor{dkgreen}{rgb}{0,0.6,0}
\definecolor{gray}{rgb}{0.5,0.5,0.5}
\definecolor{mauve}{rgb}{0.58,0,0.82}

\lstset{frame=tb,
  % language=Java,
  language=Python,
  aboveskip=3mm,
  belowskip=3mm,
  showstringspaces=false,
  columns=flexible,
  basicstyle={\small\ttfamily},
  numbers=none,
  numberstyle=\tiny\color{gray},
  keywordstyle=\color{blue},
  commentstyle=\color{dkgreen},
  stringstyle=\color{mauve},
  breaklines=true,
  breakatwhitespace=true,
  tabsize=2
}


\begin{document}




For the final project, just list as many as possible here:

- UNET Pytorch Github: \url{https://github.com/milesial/Pytorch-UNet} \cite{unet_pytorch_github}

- UNET tutorial Pytorch (and Keras as well): \cite{unet_pyimagesearch_guide}

- \url{https://scholar.google.co.uk/citations?user=fgSULeUAAAAJ&hl=en&oi=sra}

- Introduction to mathematical imaging (total variation)\cite{papafitsoros_2015}

- A COMBINED FIRST AND SECOND ORDER VARIATIONAL APPROACH FOR IMAGE
RECONSTRUCTION \cite{papafitsoros_2013}

- Book "Variational Methods in Image Processing" \cite{Gilboa2018}

- Book "Variational Methods in Imaging" \cite{var_methods_imaging}

- Introduction to CNN \cite{Lecun_CNN}

  \url{https://ieeexplore.ieee.org/document/726791}

- Introduction to UNET \cite{unet}

- UNET and TV \cite{kostas_tv_unet}

- TV in JPEG Decompression \cite{jpeg_decompress}

- Comparing TV, TV$^2$, ICTV and TGV \cite{benning_et_al}

- Benning's Thesis \cite{benning_thesis}

- Stripped-Down UNET \cite{sd_unet} \url{https://www.ncbi.nlm.nih.gov/pmc/articles/PMC7167802/pdf/diagnostics-10-00110.pdf}

- Algorithm Unrolling : Literature Review of UNETs ? (2020) \cite{algo_unrolling}
    - summarise popular techniques for algorithm unrolling in various domains of signal and image processing including imaging, vision and recognition, and speech processing. 
    - show the connections between iterative algorithms and neural networks and present recent theoretical results. 
    - discuss limitations of unrolling and suggest possible future research directions.

- \cite{Guo_2019}

- \cite{nn_inverse_problems}

- Le Cun came up with UNET? \cite{fast_approx_sparse_coding}

- Thomas Pock explains Total Variant \cite{thomas_pock_TV}

- Other Image Processing Methods

    - \cite{taylor_former} TaylorFormer for Image Dehazing : \url{https://arxiv.org/abs/2308.14036} \url{https://github.com/fvl2020/iccv-2023-mb-taylorformer} \url{https://paperswithcode.com/paper/mb-taylorformer-multi-branch-efficient}

\printbibliography

\end{document}

