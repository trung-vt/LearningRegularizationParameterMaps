\documentclass[12pt]{article}
\usepackage[utf8]{inputenc}
\usepackage{csquotes}
\usepackage{amssymb,tikz,pdftexcmds,xparse}
\usepackage{url}
\usepackage{titlesec}
\usepackage{biblatex}
\addbibresource{references.bib}
\usepackage{indentfirst}
\usepackage{amsmath}
\usepackage{enumitem}
\usepackage{xcolor}
\usepackage{algorithm}
\usepackage{algorithmic}
\usepackage{graphicx}
\usepackage{subcaption}
\usetikzlibrary{shapes, arrows.meta, positioning}

\titlespacing*{\section}
{0pt}{10pt}{4pt}
\titlespacing*{\subsection}
{0pt}{8pt}{3pt}
\titlespacing*{\subsubsection}
{0pt}{12pt}{1pt}

\tikzset{box/.style={
    minimum size=0.225cm,
    inner sep=0pt,
    draw,
  },  
  insert mark/.style={
    append after command={%
         node[inner sep=0pt,#1]
           at (\tikzlastnode.center){$\checkmark$}
     }     
  },
  insert bad mark/.style={
    append after command={%
         [shorten <=\pgflinewidth,shorten >=\pgflinewidth]
         (\tikzlastnode.north west)edge[#1](\tikzlastnode.south east)
         (\tikzlastnode.south west)edge[#1](\tikzlastnode.north east)
     }     
  },
}

\makeatletter
\NewDocumentCommand{\tikzcheckmark}{O{} m}{%
  \ifnum\pdf@strcmp{#2}{mark}=\z@%
    \tikz[baseline=-0.5ex]\node[box,insert mark={#1},#1]{};%
  \fi%
  \ifnum\pdf@strcmp{#2}{bad mark}=\z@%
    \tikz[baseline=-0.5ex]\node[box,insert bad mark={#1},#1]{};%
  \fi%
  \ifnum\pdf@strcmp{#2}{no mark}=\z@%
    \tikz[baseline=-0.5ex]\node[box,#1]{};%
  \fi%
}
\makeatother

\makeatletter
\renewcommand{\maketitle}{
  {\LARGE \@title \par}
  {\large \@author \par}
}
\makeatother

\usepackage[english]{babel}
\usepackage[letterpaper,top=2cm,bottom=2cm,left=2cm,right=2cm,marginparwidth=1.75cm]{geometry}
\usepackage[colorlinks=true, allcolors=blue]{hyperref}

\title{MTH786P Project - Diabetes Prediction}
\author{Thanh Trung Vu - 230849442}

\begin{document}

\maketitle

\section{Introduction}

This project is largely based on existing research on learning regularisation parameter maps for variational image reconstruction using deep neural networks and algorithm unrolling \cite{kofler2023learning}.
In this dissertation, we 
focus on the problem of image denoising.
We define solving the denoising problem as solving the total variation (TV) problem and give a mathematical formulation of variational image denoising. 
We then discuss the primal dual algorithm for solving the TV denoising problem, and the importance of choosing the regularisation parameter.
This led to the idea of using a deep learning method called U-Net to help us find a regularisation parameter map which can improve the denoising performance of the primal dual algorithm.
We combine a U-Net architecture with the primal dual algorithm to create an end-to-end unsupervised model for image denoising that can achieve said improvement while staying completely interpretable. 
We implement the proposed combined model and evaluate it on the Chest X-ray dataset.
The results show that the combined model consistently outperforms the traditional method of using a single regularisation parameter.

\printbibliography

\end{document}
